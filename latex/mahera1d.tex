\documentclass[a4paper]{article}
\usepackage{amsmath,a4wide}  % for mathematical expressions

\begin{document}

\title{1D Mahera}
\author{Mohamad Samman}
\date{24 March 2025}
\maketitle

\section{Introduction}

In this article, we will discuss an example involving one parameter.

\section{Example with One Parameter}
Consider the following system:
\begin{equation}
\dot{x} = -x + \theta \cdot u
\end{equation}
\begin{equation}
y = x + v
\end{equation}

Where:
\begin{itemize}
    \item \( \dot{x} \) represents the time derivative of \( x \),
    \item \( \theta \) is a constant parameter,
    \item \( u \) is an input, and
    \item \( v \) is a gaussian white noise with \(E(v)=0\) and \(E(v(t)v(\tau))=\sigma^2\delta(t-\tau)\).
\end{itemize}
The Information matrix can be shown to be:
\begin{equation}
M = \int_0^T \frac{1}{\sigma^2} x_{\theta}^2 \, d\theta
\end{equation}

Where:
\begin{itemize}
    \item \( x_{\theta} \) is the derivative of \( x \) with respect to \( \theta \),
    \item \( T \) is the time horizon,
    \item \( \sigma^2 \) is the variance of the noise process \( v \).
\end{itemize}
Let the input be energy constrained as follows:
\begin{equation}
    \int_0^T u^2 dt = E
    \label{nrjconst}
\end{equation}
The sensitivity function $\dot{x_\theta}$ is obtained as follows:
\begin{equation}
  \dot{x_\theta}  = -x_\theta + u
\end{equation}

The maximization of \( M \) is subject to a constraint \eqref{nrjconst} is equivalent to the minimization of the performance index:
\begin{equation}
J = -\frac{1}{2} \int_0^T \dfrac{-x_\theta^2}{\sigma^2} + (u^2 - \dfrac{E}{T} )dt
\end{equation}
The Pontryagin Maximum Principle (PMP) gives us the Hamiltonian:
\begin{equation}
\mathcal{H} = \dfrac{1}{2} \left[\dfrac{-x_\theta^2}{\sigma^2} + (u^2 - \dfrac{E}{T} )\right] + \lambda[-x_\theta + u]
\label{hamiltonian}
\end{equation}
Where $\lambda$ is the costate scalar\\
\[
    \dot{\lambda} = -\dfrac{\partial \mathcal{H}}{\partial x_\theta}
\]
\\
\begin{equation}
    \dot{\lambda} = -\lambda + \dfrac{-x_\theta}{\sigma^2}
    \label{labdadot}
\end{equation}
Maximization Condition:
\[\mathcal{H}_u=0\]
or:
\begin{equation}
    u^\star = - \dfrac{1}{\mu}\lambda
\end{equation}
The boundary conditions are homogeneous.
\[
x_\theta(0)=0,\quad \lambda(T)=0
\]
Substituting for $u^\stqr$ in \eqref{hamiltonian}, we obtain the two-point boundary value problem:
\begin{equation}
    \frac{d}{dt} \left[ \begin{array}{c} x_\theta \\ \lambda \end{array} \right]
= \left[ \begin{array}{cc} -1 & -\frac{1}{\mu} \\ \frac{1}{\sigma^2} & 1 \end{array} \right]
\left[ \begin{array}{c} x_\theta \\ \lambda \end{array} \right]
\label{matsys}
\end{equation}
Let $\Phi(T,0;\mu)$ be the transition matrix of \eqref{matsys} for a particular $\mu$
\begin{equation}
     \left[ \begin{array}{c} x_\theta(T) \\ \lambda(T) \end{array} \right]
= \left[ \begin{array}{cc} \Phi_{xx}(T,0;\mu) & \Phi_{x\lambda}(T,0;\mu) \\ \Phi_{\lambda x}(T,0;\mu) & \Phi_{\lambda\lambda}(T,0;\mu) \end{array} \right]
\left[ \begin{array}{c} x_\theta(0) \\ \lambda(0) \end{array} \right]
\label{matexp}
\end{equation}
The second equation in \eqref{matexp} and the boundary conditions gives
\begin{equation}
    \lambda(T)= \Phi_{\lambda\lambda}(T,0;\mu)\lambda(0)=0
    \label{eq2matexp}
\end{equation}
For a non trivial solution
\begin{equation}
    |\Phi_{\lambda\lambda}(T,0;\mu)|=0
    \label{eveq}
\end{equation}
Equation \eqref{eveq} is the eigenvalue equation for the Hamiltonian system \eqref{hamiltonian}. It can be solved by a Newton-Raphson iteration.\\
The Ricatti 
\end{document}

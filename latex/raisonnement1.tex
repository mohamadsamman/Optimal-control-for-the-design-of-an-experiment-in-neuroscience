\documentclass[a4paper]{article}
\usepackage{amsmath, amssymb, amsthm}

% Title and author
\title{Mathematical Reasoning Notes}
\author{Mohamad SAMMAN}
\date{March 2025}

% Theorem, Definition, and Proof environments
\newtheorem{theorem}{Theorem}
\newtheorem{lemma}{Lemma}
\newtheorem{corollary}{Corollary}
\theoremstyle{definition}
\newtheorem{definition}{Definition}
\theoremstyle{remark}
\newtheorem{remark}{Remark}

\begin{document}

\maketitle

\section{Problème de Contrôle Non Linéaire Paramétré et Observé}

Considérons un système dynamique non linéaire paramétré de la forme :
\begin{equation}
    \begin{cases}
        \dot{x} = f(x, u, \theta), \\
        y = h(x),
    \end{cases}
\end{equation}
avec $ x(0) $ fixé et où :
\begin{itemize}
    \item $ x \in \mathbb{R}^n $ est l'état du système,
    \item $ u \in \mathbb{R}^m $ est l'entrée de commande,
    \item $ \theta \in \mathbb{R}^p $ est un paramètre inconnu,
    \item $ y \in \mathbb{R}^q $ est la sortie idéale supposée connue,
    \item $ f $ et $ h $ sont des fonctions différentiables appropriées.
\end{itemize}

Cependant, dans un contexte réel, la sortie mesurée est affectée par un bruit $ w(t) $, ce qui donne :
\begin{equation}
    y_{\text{mes}} = h(x(t)) + w(t),
\end{equation}
où $ w(t) $ représente un bruit d'observation.

\section{Sensibilité aux Paramètres}

Nous définissons la sensibilité du système par rapport aux paramètres $\theta$ à travers la variable $ x_{\theta} = \dfrac{\partial x}{\partial \theta} $, qui satisfait le système :
\begin{equation}
    \begin{cases}
        \dot{x}_{\theta} = \frac{\partial f}{\partial x} (x, u, \theta) x_{\theta} + \frac{\partial f}{\partial \theta} (x, u, \theta), \\
        y_{\theta} = \frac{\partial h}{\partial \theta} (x) = \frac{\partial h}{\partial x} (x) x_{\theta}.
    \end{cases}
\end{equation}

\begin{remark}
    Ici, $ y_{\theta} = \frac{\partial y}{\partial \theta} $ représente la sensibilité de la sortie par rapport aux paramètres.
\end{remark}

\section{Quantification de l'écart entre \( y \) et \( y_{\text{mes}} \)}

Nous souhaitons quantifier l'écart entre la sortie idéale \( y(t) \) et la sortie mesurée \( y_{\text{mes}}(t) \) à travers :

\begin{equation}
J(\theta, \theta^\star) = \int_0^T \left| y(t) - y_{\text{mes}}(t) \right|^2 dt
\end{equation}

où :
\begin{itemize}
    \item \( y(t) \) est la "vraie" sortie du système,
    \item \( y_{\text{mes}}(t) = h(x(t)) + w(t) \) est la sortie mesurée, affectée par le bruit \( w(t) \).
\end{itemize}
\\
Notons $y\star := h(x)$ alors \( J(\theta, \theta^\star) \) devient : 
\begin{equation}
J(\theta, \theta^\star) = \int_0^T \left| y(t) - y^\star(t) - w(t) \right|^2 dt
\label{Jdep}
\end{equation}
\\
En développant le carré et en utilisant la linéarité de l'intégrale:
\begin{equation}
J(\theta, \theta^\star) = \int_0^T |y(t) - y^\star(t)|^2 dt  
- 2 \int_0^T  (y(t) - y^\star(t))^T w(t) dt  
+ \int_0^T |w(t)|^2 dt
\label{Jdev}
\end{equation}
\\
Supposons que la différence entre les paramètres estimés et les vrais paramètres \( \theta - \theta^\star \) est petite. Nous pouvons alors développer \( y - y^\star \) en série de Taylor autour de \( \theta^\star \), ce qui donne l'approximation suivante :

\begin{equation}
y - y^\star \approx (\theta - \theta^\star)^T y_{\theta^\star} + \frac{1}{2} (\theta - \theta^\star)^T R (\theta - \theta^\star)
\end{equation}

où :
\begin{itemize}
    \item \(  y_{\theta^\star} = \frac{\partial y}{\partial \theta} |_{ \theta^\star} \) représente la sensibilité de la sortie par rapport aux paramètres,
    \item \( R \) est une matrice qui modélise un terme quadratique,
\end{itemize}
\\
Nous pouvons négliger les terme d'ordre 2 et substituer dans (\ref{Jdev}) :
\begin{equation}
\begin{aligned}
J(\theta, \theta^\star) = & (\theta - \theta^\star)^T \left( \int_0^T y_{\theta^\star}(t) y_{\theta^\star}(t)^T dt \right) (\theta - \theta^\star) \\
& \quad + 2 \left(\int_0^T w(t)^T y_{\theta^\star}(t) dt \right) (\theta - \theta^\star) \\
& \quad + \int_0^T |w(t)|^2 dt + R'
\end{aligned}
\end{equation}
\\
Notons: $M(\theta^\star) = \int_0^T y_{\theta^\star}(t) y_{\theta^\star}(t)^T dt$ et $V(\theta^\star) = \int_0^T w(t)^T y_{\theta^\star}(t) dt.$
\end{equation}
Alors on obtient:
\begin{equation}
\begin{aligned}
J(\theta, \theta^\star) = & (\theta - \theta^\star)^T M(\theta^\star) (\theta - \theta^\star) + 2 V(\theta^\star) (\theta - \theta^\star) + \int_0^T |w(t)|^2 dt + R'.
\end{aligned}
\end{equation}
\\
Le minimum de $J$ est atteint pour $\theta$ qui vérifie:
\begin{equation}
M(\theta^\star)(\theta - \theta^\star) + V(\theta^\star) = \theta
\end{equation} 
\\
C'est à dire en $\theta$ qui vérifie:
\begin{equation}
\theta = \theta^\star - M^{-1}(\theta^\star)V(\theta^\star)
\end{equation}

\end{document}
